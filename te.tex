\documentclass[11pt,letterpaper]{article}

\usepackage[utf8]{inputenc}
\usepackage[spanish]{babel}
\usepackage{amsmath}
\usepackage{amsfonts}
\usepackage{amssymb}
\usepackage{graphicx}
\usepackage[left=2cm,right=2cm,top=2cm,bottom=2cm]{geometry}




\begin{document}
\title{Trabajo de Investigacion: Ajuste de curvas por mínimos cuadrados}

\author{
	Alberto Ramos Cruz
	\and Moises Alonso Marroquin Ayala
	\and Rene Eduardo Hernandez Castro
	\and Roberto Jose Melgares Zelaya
}
\maketitle

\section{Regresión lineal por mínimos cuadrados}
\subsection{Ajuste de una recta por regresión por mínimos cuadrados}
Con el fin de encontrar un modelo matemático que represente lo mejor posible a una serie de datos experimentales, es posible abordarlo por medio de una curva $y=\phi(x)$ que se aproxime a los datos, sin la necesidad de que esta curve pase por ellos.
Lo anterior plantea el problema de verificar que en los terminos:
$\{(x_i, f(x_i))\quad i = 1,2,3, \cdots, n\} $ se debe de hallar un $\phi(x)$ que verifique

$$Minima = \sum_{n}^{i=1} (f(x_i) - \phi(x_i))^2$$
\subsection{Linealización de relaciones no lineales}

\section{Aproximación polinomial con mínimos cuadrados}
\subsection{Parabola de mínimos cuadrados}
La parábola de mínimos cuadrados pasa a ser una curva de $y=\phi(x)$ donde $\phi(x)$ es un polinomio de segundo grado que viene dado por la ecuación
$$y = a_0 + a_1 x + a_2 x^2  $$
donde $a_0, a_1, a_2 \in R$ y se determinan resolviendo las ecuaciones simultaneas de:

$$\sum y = a_0n + a_1\sum x + a_2\sum x^2$$
$$\sum xy = a_0\sum x + a_1\sum x^2 + a_2\sum x^3$$
$$\sum x^2y = a_0\sum x^2 + a_1\sum x^3 + a_2\sum x^3$$

\subsection{Ajuste de polinomios de grado \textit{n}	 (caso general)}

\end{document}