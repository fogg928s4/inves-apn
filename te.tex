\documentclass[11pt,letterpaper]{article}

% PAQUETES ++++++++++++++++++++++++++++++++
\usepackage[utf8]{inputenc}
\usepackage[spanish]{babel}
\usepackage{amsmath}
\usepackage{amsfonts}
\usepackage{amssymb}
\usepackage[left=2cm,right=2cm,top=2cm,bottom=2cm]{geometry}
\pagenumbering{arabic}
\usepackage[pdftex]{graphicx}
\usepackage{eqparbox}
\usepackage{tabularx}
\usepackage{booktabs}
\usepackage{array}
% \usepackage{blindtext}
\usepackage{url}
\usepackage{relsize}
\usepackage{multirow}
\usepackage{float}
\usepackage{tabularray}


%+++++++++++++++++++++++++++++++++++++++++++++++++++
\graphicspath{ {./img/}}
\pagenumbering{arabic}

\begin{document}
\title{Trabajo de Investigación: Ajuste de Curvas por Mínimos Cuadrados}

\author{
	Alberto Ramos Cruz
	\and Moisés Alonso Marroquin Ayala
	\and Rene Eduardo Hernandez Castro
	\and Roberto Jose Melgares Zelaya
}
\maketitle


\section{Regresión lineal por mínimos cuadrados}
\subsection{Ajuste de una recta por regresión por mínimos cuadrados}


\subsection{Linealización de relaciones no lineales}

\section{Aproximación polinomial con mínimos cuadrados}
Con el fin de encontrar un modelo matemático que represente lo mejor posible a una serie de datos experimentales, es posible abordarlo por medio de una curva $y=\phi(x)$ que se aproxime a los datos, sin la necesidad de que esta curva pase por ellos.
Lo anterior plantea el problema de verificar que en los términos:
$\{(x_i, y_i)\quad i = 0,1,2,3, \cdots, N\} $ se debe de hallar un $\phi(x)$ que verifique

$$Minima = \sum_{i=0}^{N} (y_i - \phi(x_i))^2$$

Para evitar problemas se suele utilizar una diferencia cuadrada para evitar problemas de derivabilidad. 
\subsection{Parábola de mínimos cuadrados}
La parábola de mínimos cuadrados tiene como objetivo aproximar un conjunto de puntos 
$(x_0, y_0),$ $(x_1, y_1),$ $ (x_2, y_2),$ $ \cdots , (x_n, y_N)$, donde $N$ es el número de elementos en el conjunto de puntos, 
a través de una curva polinomial de grado 2. Entonces, podemos entender la parábola de mínimos cuadrados como una curva de $y=\phi(x)$ donde $\phi(x)$ es un polinomio de segundo grado que viene dado por la ecuación
$$y = a_0 + a_1 x + a_2 x^2  $$
donde $a_0, a_1, a_2 \in R$ y se determinan resolviendo las ecuaciones simultaneas : 

\begin{equation} \label{eq:sist-parab} 
	\left\{
		\begin{array}{@{}l@{}}
			\sum_{i=0}^{N} y_i  = a_0N + a_1\sum_{i=0}^{N} x_i + a_2\sum_{i=0}^{N} x_i^2 \cr\cr
			\sum_{i=0}^{N} x_iy = a_0\sum_{i=0}^{N} x_i + a_1\sum_{i=0}^{N} x_i^2 + a_2\sum_{i=0}^{N} x_i^3 \cr\cr
			\sum_{i=0}^{N} x_i^2y = a_0\sum_{i=0}^{N} x_i^2 + a_1\sum_{i=0}^{N} x_i^3 + a_2\sum_{i=0}^{N} x_i^4 		
		\end{array}
	\right.
\end{equation}
\linebreak 
Las ecuaciones anteriores  son conocidads como \textit{ecuaciones normales de mínimos cuadrados}. Estas ecuaciones se obtienen al multiplicar la ecuación $y = a_0 + a_1 x + a_2 x^2  $ por  1, $x$, y $x^2$, respectivamente. Como se puede observar, se obtiene un \emph{sistema con 3 incógnitas y 3 ecuaciones.}
\par
Para obtener los valores que acompañan a las incógnitas $a_i$ en el sistema de ecuaciones se debe de obtener la suma de los productos, tomando en cuenta los valores respectivos de $x_i$ con sus imágenes en $y_i$; esto a excepción del valor $N$ que acompaña a la incógnita $a_0$ en la primera ecuación como se puede observar en la Ecuación \ref{eq:sist-parab}.
\par
A continuación, se procede a resolver el sistema de ecuaciones por medio de una matriz de 4$\times$3 para que de esste modo se obtengan los valores de las 3 incógnitas. En este caso, se puede usar cualquier método de resolución de sistemas de ecuaciones por medio de matrices, aunque para esta investigación se hará uso del método de Cramer. \emph{ Con el objetivo de mejorar la comprensión lectora, a partir de este punto se usará la siguiente notación.}

 \begin{table}[!ht]
 \centering
	\begin{tabular}{ c c }
	\hline
	Símbolo & Significado  \\ \hline
	$X$ &	$\mathlarger{\sum}_{i=0}^{N} x_i$  \\
	$Y$ &   $\mathlarger{\sum}_{i=0}^{N} y_i$ \\
	$X^2$ & $\mathlarger{\sum}_{i=0}^{N} x_i^2$ \\
	$X^3$ & $\mathlarger{\sum}_{i=0}^{N} x_i^3$ \\ 
	$X^4$ & $\mathlarger{\sum}_{i=0}^{N} x_i^4$ \\
	$XY$ &  $\mathlarger{\sum}_{i=0}^{N} x_i y_i$	\\
	$X^2Y$& $\mathlarger{\sum}_{i=0}^{N} x^2_i y_i$	\\ \hline
	\end{tabular}
	\label{table:simbologia}
\end{table}

Usando la notación anterior, pasamos a resolver la matriz y obtener las ecuaciones que nos dan los determinantes. \emph{Recordar que estas matrices no contienen variables simbólicas, sino que valores escalares representados con letras mayúsculas.}
\begin{table}[H] \centering \begin{tabular}{c c c c c}
%% resolucion a0
$a_0 = \frac{
		\begin{vmatrix}
		 	Y &	X	& X^2 \\
		 	XY &	X^2	& X^3 \\
		 	X^2Y & X^3 & X^4	 
		\end{vmatrix}}
		{\begin{vmatrix}
		 	N &	X	& X^2 \\
		 	X &	X^2	& X^3 \\
		 	X^2 & X^3 & X^4	 
		\end{vmatrix}}
		
$ & &
% resolucion a1
$ a_1 = \frac{
		\begin{vmatrix}
		 	N &	Y	& X^2 \\
		 	X &	XY	& X^3 \\
		 	X^2 & X^2Y & X^4 
		\end{vmatrix}}
		{\begin{vmatrix}
		 	N &	X	& X^2 \\
		 	X &	X^2	& X^3 \\
		 	X^2 & X^3 & X^4	 
		\end{vmatrix}}
$ & &
% resolucion a2
$ a_2 = \frac{
		\begin{vmatrix}
		 	N &	X	& Y \\
		 	X &	X^2	& XY \\
		 	X^2 & X^3 & X^2Y 
		\end{vmatrix}}
		{\begin{vmatrix}
		 	N &	X	& X^2 \\
		 	X &	X^2	& X^3 \\
		 	X^2 & X^3 & X^4	 
		\end{vmatrix}}
$ \\
\end{tabular} \end{table}


\subsection{Ajuste de polinomios de grado \textit{n}	 (caso general)}
aserejejadejedejedetudejedeseinoubadajabeedebugidiji

\bibliographystyle{IEEEtran}
\bibliography{bibi}
\end{document}